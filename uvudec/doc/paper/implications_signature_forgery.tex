Collisions in cryptographic hashes lead to a number of interesting applications. 
Cryptographic hashes are used for verification purposes to establish trust, authority,
and authenticity of information. With collisions these things fall apart. 

There are various degrees of attacks. The most severe is a preimage attack, where
a known hash is collided against. This represents a total compromise where information
can be arbitrarily modified in the target document. Rather dramatically, this past December
researchers demonstrated that they were able to forge SSL certificates
\cite{rogue_ca} to bypass the entire SSL PKI infrastructure.

The second class of collisions is known as a second preimage, where colliding hashes
are generated. In block-based hashing systems such as MD5 it is possible to use second 
pre-image attacks to produce documents and programs with different apperances and behaviors
that still maintain the same cryptographic signatures. The only differences between them
are the colliding blocks. It is these colliding blocks that determine if the program should
behave in one way or another.

A possible scenario is providing documents in electronic form, PDF for instance. A contract
is drafted by Mallory. Mallory produces two versions. The first appears legitimate.
The second preimage Mallory keeps in secret, and contains information that would put Mallory
to some advantage. Alice is given the legitimate version of the contract by Mallory. Alice
agrees with the contract and signs the document based on a cryptographic signature of the version
she is presented. Unfortunately for Alice, her signature is valid for both the contract she read
and the one Mallory produced in secret. Mallory can now unfairly take advantage of Alice.

The above is an unpractical application. The real danger lies with automated systems that rely upon
PKI such as the MD5 attack earlier described.

The birthday attack recognizes that simple mathematical probability can be applied to naively produce
second preimages without any weaknesses in the hash. A 64-bit signature, although not seriously considired 
today, provides a perfect target for parallelization. Although a single machine still struggles
with the massive amount of information, given 128 machines.